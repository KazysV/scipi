\hspace{1 em} The technologies for aggregating passenger flow data have had significant
advancements in the last decade. With the development of the Internet of Things (IoT)
devices, Intelligent Transport Systems (ITS) have become widely available for public
transportation fleet operators. Smart ticket composters, GPS tracking devices, automated
passenger counters (APC) and data transmission technologies such as 5G networks became
common with monitoring and managing public transport. 
The collected data can be used not only for tracking Key Performance Indicators (KPIs),
but also to model passenger flow and predict future demand. 

The passenger flow prediction could be divided in terms of prediction time scale: short-term
forecasting (minutes to hours), medium-term forecasting (hours to days) and long-term (months, seasons of the year).
For short term passenger flow forecasting, short term traffic flow forecasts have an effect. Moreover,
short term flow prediction could be divided in parametric and non-parametric models. Parametric models

Parametric models is the standard approach for short-term flow forecasting. Models such as 
Seasonal Autoregresive Integrated Moving Average (SARIMA) and Autoregressive Integrated Moving Average (ARIMA)
have been used to predict demands for fleet\cite{ARIMA}. These models are well suited for time series predictions
and widely used in the field of transportation management. However, due to linear nature of these models, they are not
well suited to capture the non-linear patterns in the data.

Non-parametric approach such as Long Short-Term Memory (LSTM) reccurent neural networks (RNN) have been used to predict passenger flow\cite{LSTM}.
However RNN models have a flaw of vanishing gradient problem, which makes it difficult to train the model for long sequences of data.
To overcome this problem, LSTM models were introduced. LSTM models have a memory cell that can store information for long periods of time.
